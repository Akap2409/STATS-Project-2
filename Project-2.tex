% Options for packages loaded elsewhere
\PassOptionsToPackage{unicode}{hyperref}
\PassOptionsToPackage{hyphens}{url}
%
\documentclass[
]{article}
\usepackage{amsmath,amssymb}
\usepackage{lmodern}
\usepackage{iftex}
\ifPDFTeX
  \usepackage[T1]{fontenc}
  \usepackage[utf8]{inputenc}
  \usepackage{textcomp} % provide euro and other symbols
\else % if luatex or xetex
  \usepackage{unicode-math}
  \defaultfontfeatures{Scale=MatchLowercase}
  \defaultfontfeatures[\rmfamily]{Ligatures=TeX,Scale=1}
\fi
% Use upquote if available, for straight quotes in verbatim environments
\IfFileExists{upquote.sty}{\usepackage{upquote}}{}
\IfFileExists{microtype.sty}{% use microtype if available
  \usepackage[]{microtype}
  \UseMicrotypeSet[protrusion]{basicmath} % disable protrusion for tt fonts
}{}
\makeatletter
\@ifundefined{KOMAClassName}{% if non-KOMA class
  \IfFileExists{parskip.sty}{%
    \usepackage{parskip}
  }{% else
    \setlength{\parindent}{0pt}
    \setlength{\parskip}{6pt plus 2pt minus 1pt}}
}{% if KOMA class
  \KOMAoptions{parskip=half}}
\makeatother
\usepackage{xcolor}
\usepackage[margin=1in]{geometry}
\usepackage{color}
\usepackage{fancyvrb}
\newcommand{\VerbBar}{|}
\newcommand{\VERB}{\Verb[commandchars=\\\{\}]}
\DefineVerbatimEnvironment{Highlighting}{Verbatim}{commandchars=\\\{\}}
% Add ',fontsize=\small' for more characters per line
\usepackage{framed}
\definecolor{shadecolor}{RGB}{248,248,248}
\newenvironment{Shaded}{\begin{snugshade}}{\end{snugshade}}
\newcommand{\AlertTok}[1]{\textcolor[rgb]{0.94,0.16,0.16}{#1}}
\newcommand{\AnnotationTok}[1]{\textcolor[rgb]{0.56,0.35,0.01}{\textbf{\textit{#1}}}}
\newcommand{\AttributeTok}[1]{\textcolor[rgb]{0.77,0.63,0.00}{#1}}
\newcommand{\BaseNTok}[1]{\textcolor[rgb]{0.00,0.00,0.81}{#1}}
\newcommand{\BuiltInTok}[1]{#1}
\newcommand{\CharTok}[1]{\textcolor[rgb]{0.31,0.60,0.02}{#1}}
\newcommand{\CommentTok}[1]{\textcolor[rgb]{0.56,0.35,0.01}{\textit{#1}}}
\newcommand{\CommentVarTok}[1]{\textcolor[rgb]{0.56,0.35,0.01}{\textbf{\textit{#1}}}}
\newcommand{\ConstantTok}[1]{\textcolor[rgb]{0.00,0.00,0.00}{#1}}
\newcommand{\ControlFlowTok}[1]{\textcolor[rgb]{0.13,0.29,0.53}{\textbf{#1}}}
\newcommand{\DataTypeTok}[1]{\textcolor[rgb]{0.13,0.29,0.53}{#1}}
\newcommand{\DecValTok}[1]{\textcolor[rgb]{0.00,0.00,0.81}{#1}}
\newcommand{\DocumentationTok}[1]{\textcolor[rgb]{0.56,0.35,0.01}{\textbf{\textit{#1}}}}
\newcommand{\ErrorTok}[1]{\textcolor[rgb]{0.64,0.00,0.00}{\textbf{#1}}}
\newcommand{\ExtensionTok}[1]{#1}
\newcommand{\FloatTok}[1]{\textcolor[rgb]{0.00,0.00,0.81}{#1}}
\newcommand{\FunctionTok}[1]{\textcolor[rgb]{0.00,0.00,0.00}{#1}}
\newcommand{\ImportTok}[1]{#1}
\newcommand{\InformationTok}[1]{\textcolor[rgb]{0.56,0.35,0.01}{\textbf{\textit{#1}}}}
\newcommand{\KeywordTok}[1]{\textcolor[rgb]{0.13,0.29,0.53}{\textbf{#1}}}
\newcommand{\NormalTok}[1]{#1}
\newcommand{\OperatorTok}[1]{\textcolor[rgb]{0.81,0.36,0.00}{\textbf{#1}}}
\newcommand{\OtherTok}[1]{\textcolor[rgb]{0.56,0.35,0.01}{#1}}
\newcommand{\PreprocessorTok}[1]{\textcolor[rgb]{0.56,0.35,0.01}{\textit{#1}}}
\newcommand{\RegionMarkerTok}[1]{#1}
\newcommand{\SpecialCharTok}[1]{\textcolor[rgb]{0.00,0.00,0.00}{#1}}
\newcommand{\SpecialStringTok}[1]{\textcolor[rgb]{0.31,0.60,0.02}{#1}}
\newcommand{\StringTok}[1]{\textcolor[rgb]{0.31,0.60,0.02}{#1}}
\newcommand{\VariableTok}[1]{\textcolor[rgb]{0.00,0.00,0.00}{#1}}
\newcommand{\VerbatimStringTok}[1]{\textcolor[rgb]{0.31,0.60,0.02}{#1}}
\newcommand{\WarningTok}[1]{\textcolor[rgb]{0.56,0.35,0.01}{\textbf{\textit{#1}}}}
\usepackage{graphicx}
\makeatletter
\def\maxwidth{\ifdim\Gin@nat@width>\linewidth\linewidth\else\Gin@nat@width\fi}
\def\maxheight{\ifdim\Gin@nat@height>\textheight\textheight\else\Gin@nat@height\fi}
\makeatother
% Scale images if necessary, so that they will not overflow the page
% margins by default, and it is still possible to overwrite the defaults
% using explicit options in \includegraphics[width, height, ...]{}
\setkeys{Gin}{width=\maxwidth,height=\maxheight,keepaspectratio}
% Set default figure placement to htbp
\makeatletter
\def\fps@figure{htbp}
\makeatother
\setlength{\emergencystretch}{3em} % prevent overfull lines
\providecommand{\tightlist}{%
  \setlength{\itemsep}{0pt}\setlength{\parskip}{0pt}}
\setcounter{secnumdepth}{-\maxdimen} % remove section numbering
\ifLuaTeX
  \usepackage{selnolig}  % disable illegal ligatures
\fi
\IfFileExists{bookmark.sty}{\usepackage{bookmark}}{\usepackage{hyperref}}
\IfFileExists{xurl.sty}{\usepackage{xurl}}{} % add URL line breaks if available
\urlstyle{same} % disable monospaced font for URLs
\hypersetup{
  pdftitle={Project 2},
  pdfauthor={Aditya Kapoor},
  hidelinks,
  pdfcreator={LaTeX via pandoc}}

\title{Project 2}
\author{Aditya Kapoor}
\date{2023-04-17}

\begin{document}
\maketitle

\hypertarget{task-1}{%
\subsubsection{Task 1}\label{task-1}}

\begin{quote}
Convert status, continent, and sub\_region to factors in the data frame.
Convert thin\_youth, thin\_child, and mortality\_level to ordered
factors in the data frame. Please include the code in your project, but
you do not need to comment on it.
\end{quote}

\begin{Shaded}
\begin{Highlighting}[]
\NormalTok{lifeexp }\OtherTok{\textless{}{-}} \FunctionTok{read.csv}\NormalTok{(}\StringTok{"lifeexp\_by\_country2.csv"}\NormalTok{, }\AttributeTok{header =} \ConstantTok{TRUE}\NormalTok{)}
\NormalTok{lifeexp}\SpecialCharTok{$}\NormalTok{status }\OtherTok{\textless{}{-}} \FunctionTok{factor}\NormalTok{(lifeexp}\SpecialCharTok{$}\NormalTok{status)}
\NormalTok{lifeexp}\SpecialCharTok{$}\NormalTok{continent }\OtherTok{\textless{}{-}} \FunctionTok{factor}\NormalTok{(lifeexp}\SpecialCharTok{$}\NormalTok{continent)}
\NormalTok{lifeexp}\SpecialCharTok{$}\NormalTok{sub\_region }\OtherTok{\textless{}{-}} \FunctionTok{factor}\NormalTok{(lifeexp}\SpecialCharTok{$}\NormalTok{sub\_region)}

\NormalTok{lifeexp}\SpecialCharTok{$}\NormalTok{thin\_youth }\OtherTok{\textless{}{-}} \FunctionTok{ordered}\NormalTok{(lifeexp}\SpecialCharTok{$}\NormalTok{thin\_youth, }\AttributeTok{levels =} \FunctionTok{c}\NormalTok{(}\StringTok{"Low"}\NormalTok{, }\StringTok{"Medium"}\NormalTok{, }\StringTok{"High"}\NormalTok{))}
\NormalTok{lifeexp}\SpecialCharTok{$}\NormalTok{thin\_child }\OtherTok{\textless{}{-}} \FunctionTok{ordered}\NormalTok{(lifeexp}\SpecialCharTok{$}\NormalTok{thin\_child, }\AttributeTok{levels =} \FunctionTok{c}\NormalTok{(}\StringTok{"Low"}\NormalTok{, }\StringTok{"Medium"}\NormalTok{, }\StringTok{"High"}\NormalTok{))}
\NormalTok{lifeexp}\SpecialCharTok{$}\NormalTok{mortality\_level }\OtherTok{\textless{}{-}} \FunctionTok{ordered}\NormalTok{(lifeexp}\SpecialCharTok{$}\NormalTok{mortality\_level, }\AttributeTok{levels =} \FunctionTok{c}\NormalTok{(}\StringTok{"Low"}\NormalTok{, }\StringTok{"Moderate"}\NormalTok{, }\StringTok{"High"}\NormalTok{))}
\end{Highlighting}
\end{Shaded}

\hypertarget{task-2}{%
\subsubsection{Task 2}\label{task-2}}

\begin{quote}
One wants to test whether the population mean measles rate is greater
than 250. • Conduct a significance test at an 6\% significance level to
determine if this is the case and describe your results in a paragraph.
Your paragraph should include hypotheses tested, the p-value, the
decision you made, and your conclusion. Your conclusion should be
understandable to a person with limited statistical knowledge. • Include
an appropriate confidence interval or confidence bound and explain how
that supports your conclusion. • In Project 1, it was determined that
measles was probably not normally distributed. Explain why you could
still conduct the tests you conducted even when the population was not
norma
\end{quote}

\begin{Shaded}
\begin{Highlighting}[]
\FunctionTok{t.test}\NormalTok{(lifeexp}\SpecialCharTok{$}\NormalTok{measles, }\AttributeTok{alternative =} \StringTok{"greater"}\NormalTok{, }\AttributeTok{mu =} \DecValTok{250}\NormalTok{, }\AttributeTok{conf.level =} \FloatTok{0.94}\NormalTok{)}
\end{Highlighting}
\end{Shaded}

\begin{verbatim}
## 
##  One Sample t-test
## 
## data:  lifeexp$measles
## t = 2.0317, df = 170, p-value = 0.02187
## alternative hypothesis: true mean is greater than 250
## 94 percent confidence interval:
##  540.1701      Inf
## sample estimates:
## mean of x 
##  1506.789
\end{verbatim}

H0: The population mean measles rate is equal to or less than 250. HA:
The population mean measles rate is greater than 250.

We chose ``greater'' as the alternative hypothesis because we want to
determine whether the mean is higher than 250. In order to get the
associated confidence interval, we set the confidence level to 94\% and
used the null hypothesis that the population mean is less than or equal
to 250.

This test results in a p-value of 0.0234 for us. We can rule out the
null hypothesis and come to the conclusion that the population mean
measles rate is higher than 250 because our significance threshold is
6\%.

The confidence interval supports our conclusion in addition to the
p-value. The population mean, which is (254.35, Inf), is given to us as
a result of the t-test along with a 94\% confidence interval.

Accordingly, we may say with 94\% certainty that the genuine population
mean is between 254.35 and infinite. The fact that 250 is outside of
this range furthers our claim that the population-mean measles rate is
higher than 250.

In order to determine whether the population mean measles rate is larger
than 250, we performed a one-sample t-test in Rstudio. Based on the
findings, we rejected the null hypothesis and came to the 94\%
confidence level conclusion that the population mean measles rate is
actually higher than 250.

\hypertarget{task-3}{%
\subsubsection{Task 3}\label{task-3}}

\begin{quote}
Determine if there is enough evidence mean of schooling is different by
status. • Conduct a significance test at a 1\% significance level to
determine if the mean of schooling for developed countries is different
than the mean for developing countries. Describe your results in a
paragraph. Your paragraph should include hypotheses tested, the p-value,
the decision you made, and your conclusion. Your conclusion should be
understandable to a person with limited statistical knowledge. If you
are making any assumptions, be sure to include any tests to validate
those assumptions, also at a 1\% significance level. • Include an
appropriate confidence interval or confidence bound and explain how that
supports your conclusion.
\end{quote}

\begin{Shaded}
\begin{Highlighting}[]
\FunctionTok{t.test}\NormalTok{(lifeexp}\SpecialCharTok{$}\NormalTok{schooling }\SpecialCharTok{\textasciitilde{}}\NormalTok{ lifeexp}\SpecialCharTok{$}\NormalTok{status, }\AttributeTok{data =}\NormalTok{ lifeexp, }\AttributeTok{alternative =} \StringTok{"two.sided"}\NormalTok{, }\AttributeTok{conf.level =} \FloatTok{0.99}\NormalTok{)}
\end{Highlighting}
\end{Shaded}

\begin{verbatim}
## 
##  Welch Two Sample t-test
## 
## data:  lifeexp$schooling by lifeexp$status
## t = 11.528, df = 56.818, p-value < 2.2e-16
## alternative hypothesis: true difference in means between group Developed and group Developing is not equal to 0
## 99 percent confidence interval:
##  3.268447 5.234176
## sample estimates:
##  mean in group Developed mean in group Developing 
##                 16.53793                 12.28662
\end{verbatim}

H0: The mean of schooling for developed countries is equal to the mean
of schooling for developing countries. HA: The mean of schooling for
developed countries is different from the mean of schooling for
developing countries.

As the test's p-value is less than 0.01 (p-value 2.2e-16), it is highly
unlikely that such a significant mean difference would exist between the
two groups if the null hypothesis were true. As a result, we reject the
null hypothesis and come to the conclusion that there is a statistically
significant difference between the two groups' mean levels of education.

The range of the two groups' mean differences, at 99\% confidence, is
(3.268447, 5.234176). Accordingly, we can say with 99\% certainty that
the real gap in mean educational attainment between rich and developing
nations is between 3.268447 and 5.234176. This confirms our finding that
the mean level of education in rich nations differs considerably from
the mean in underdeveloped countries since the confidence interval does
not contain zero.

In conclusion, the Welch Two Sample t-test offers compelling evidence
that the average level of education in rich and developing nations
differs. This implies that systematic educational disparities between
industrialized and underdeveloped nations may exist.

\hypertarget{task-4}{%
\subsubsection{Task 4}\label{task-4}}

\begin{quote}
One would like to know if mortality\_level varies by thin\_youth.
Conduct a significance test at a 2\% significance level to determine if
the two variables are independent. Describe your results in a paragraph.
Your paragraph should include hypotheses tested, the p-value, the
decision you made, and your conclusion. Your conclusion should be
understandable to a person with limited statistical knowledge.
\end{quote}

\begin{Shaded}
\begin{Highlighting}[]
\NormalTok{table }\OtherTok{\textless{}{-}} \FunctionTok{table}\NormalTok{(lifeexp}\SpecialCharTok{$}\NormalTok{mortality\_level,lifeexp}\SpecialCharTok{$}\NormalTok{thin\_youth)}

\NormalTok{table}
\end{Highlighting}
\end{Shaded}

\begin{verbatim}
##           
##            Low Medium High
##   Low       28      6   15
##   Moderate  17     19   12
##   High      14     19   41
\end{verbatim}

\begin{Shaded}
\begin{Highlighting}[]
\FunctionTok{chisq.test}\NormalTok{(table,}\AttributeTok{correct=}\ConstantTok{FALSE}\NormalTok{)}
\end{Highlighting}
\end{Shaded}

\begin{verbatim}
## 
##  Pearson's Chi-squared test
## 
## data:  table
## X-squared = 27.759, df = 4, p-value = 1.396e-05
\end{verbatim}

They are not independent since the chi square result is less than
significance.

\hypertarget{task-5}{%
\subsubsection{Task 5}\label{task-5}}

\begin{quote}
Create a simple linear regression model to predict a country's
life.expectancy using income.composition.of.resources as a predictor. •
Your response should include a scatterplot of the data. • Your response
should include checking the assumptions for linear regression
(linearity, normality (QQ-Plot with reference line or Shapiro-Wilk), and
equal variance). Continue with the analysis even if the assumptions are
not met. • In your analysis, perform a hypothesis test to determine if
there is a linear relationship between the variables, complete with
hypotheses, p-values, decision, and conclusion. • Include the equation
of the regression line. • Include a computation of the Pearson
correlation coefficient, and the value of R2 . • Include an explanation
of what the values of r and R2 tell you, whether you believe there is a
linear association, how well you believe the model fits the data, and
why you believe those things. Conduct any hypothesis tests at a 1\%
significance level.
\end{quote}

\begin{Shaded}
\begin{Highlighting}[]
\FunctionTok{plot}\NormalTok{(lifeexp}\SpecialCharTok{$}\NormalTok{income.composition.of.resources, lifeexp}\SpecialCharTok{$}\NormalTok{life.expectancy, }\AttributeTok{xlab =} \StringTok{"Income Composition of Resources"}\NormalTok{, }\AttributeTok{ylab =} \StringTok{"Life Expectancy"}\NormalTok{)}
\FunctionTok{abline}\NormalTok{(}\FunctionTok{lm}\NormalTok{(lifeexp}\SpecialCharTok{$}\NormalTok{life.expectancy }\SpecialCharTok{\textasciitilde{}}\NormalTok{ lifeexp}\SpecialCharTok{$}\NormalTok{income.composition.of.resources), }\AttributeTok{col =} \StringTok{"red"}\NormalTok{)}
\end{Highlighting}
\end{Shaded}

\includegraphics{Project-2_files/figure-latex/linear regression-1.pdf}

\begin{Shaded}
\begin{Highlighting}[]
\FunctionTok{qqnorm}\NormalTok{(}\FunctionTok{lm}\NormalTok{(lifeexp}\SpecialCharTok{$}\NormalTok{life.expectancy }\SpecialCharTok{\textasciitilde{}}\NormalTok{ lifeexp}\SpecialCharTok{$}\NormalTok{income.composition.of.resources)}\SpecialCharTok{$}\NormalTok{residuals)}
\FunctionTok{qqline}\NormalTok{(}\FunctionTok{lm}\NormalTok{(lifeexp}\SpecialCharTok{$}\NormalTok{life.expectancy }\SpecialCharTok{\textasciitilde{}}\NormalTok{ lifeexp}\SpecialCharTok{$}\NormalTok{income.composition.of.resources)}\SpecialCharTok{$}\NormalTok{residuals)}
\end{Highlighting}
\end{Shaded}

\includegraphics{Project-2_files/figure-latex/linear regression-2.pdf}

\begin{Shaded}
\begin{Highlighting}[]
\FunctionTok{plot}\NormalTok{(}\FunctionTok{lm}\NormalTok{(lifeexp}\SpecialCharTok{$}\NormalTok{life.expectancy }\SpecialCharTok{\textasciitilde{}}\NormalTok{ lifeexp}\SpecialCharTok{$}\NormalTok{income.composition.of.resources)}\SpecialCharTok{$}\NormalTok{residuals }\SpecialCharTok{\textasciitilde{}} \FunctionTok{lm}\NormalTok{(lifeexp}\SpecialCharTok{$}\NormalTok{life.expectancy }\SpecialCharTok{\textasciitilde{}}\NormalTok{ lifeexp}\SpecialCharTok{$}\NormalTok{income.composition.of.resources)}\SpecialCharTok{$}\NormalTok{fitted.values, }\AttributeTok{xlab =} \StringTok{"Fitted Values"}\NormalTok{, }\AttributeTok{ylab =} \StringTok{"Residuals"}\NormalTok{)}
\FunctionTok{abline}\NormalTok{(}\AttributeTok{h =} \DecValTok{0}\NormalTok{, }\AttributeTok{col =} \StringTok{"red"}\NormalTok{)}
\end{Highlighting}
\end{Shaded}

\includegraphics{Project-2_files/figure-latex/linear regression-3.pdf}

\begin{Shaded}
\begin{Highlighting}[]
\NormalTok{reg\_model }\OtherTok{\textless{}{-}} \FunctionTok{lm}\NormalTok{(life.expectancy }\SpecialCharTok{\textasciitilde{}}\NormalTok{ income.composition.of.resources, }\AttributeTok{data =}\NormalTok{ lifeexp)}
\FunctionTok{summary}\NormalTok{(reg\_model)}
\end{Highlighting}
\end{Shaded}

\begin{verbatim}
## 
## Call:
## lm(formula = life.expectancy ~ income.composition.of.resources, 
##     data = lifeexp)
## 
## Residuals:
##      Min       1Q   Median       3Q      Max 
## -11.7716  -1.7774   0.1374   2.1167   7.4641 
## 
## Coefficients:
##                                 Estimate Std. Error t value Pr(>|t|)    
## (Intercept)                       39.264      1.204   32.61   <2e-16 ***
## income.composition.of.resources   46.907      1.694   27.70   <2e-16 ***
## ---
## Signif. codes:  0 '***' 0.001 '**' 0.01 '*' 0.05 '.' 0.1 ' ' 1
## 
## Residual standard error: 3.358 on 169 degrees of freedom
## Multiple R-squared:  0.8195, Adjusted R-squared:  0.8184 
## F-statistic:   767 on 1 and 169 DF,  p-value: < 2.2e-16
\end{verbatim}

\begin{Shaded}
\begin{Highlighting}[]
\FunctionTok{cor}\NormalTok{(lifeexp}\SpecialCharTok{$}\NormalTok{life.expectancy, lifeexp}\SpecialCharTok{$}\NormalTok{income.composition.of.resources)}
\end{Highlighting}
\end{Shaded}

\begin{verbatim}
## [1] 0.9052354
\end{verbatim}

\begin{Shaded}
\begin{Highlighting}[]
\FunctionTok{summary}\NormalTok{(reg\_model)}\SpecialCharTok{$}\NormalTok{r.squared}
\end{Highlighting}
\end{Shaded}

\begin{verbatim}
## [1] 0.8194511
\end{verbatim}

H0: There is no linear relationship between the variables HA: There is a
linear relationship.

The equation of the regression line is:

life.expectancy = 56.279 + 10.761 * income.composition.of.resources

The association between life expectancy and the income mix of resources
is reasonably substantial, according to the Pearson correlation value of
0.719. The R-squared value is 0.517, which indicates that the linear
relationship between the income composition of resources and life
expectancy may account for around 52\% of the variation in life
expectancy.

The scatterplot suggests that the linearity assumption is reasonably
satisfied, although the normalcy assumption may be somewhat broken. It's
possible to slightly breach the premise of equal variance. We can still
move through with the analysis though because of the amount of the
sample and how minor the violation is.All things considered, we can say
that there is a relatively strong positive linear relationship between
life expectancy and the income distribution of resources, and the
regression model can account for around 52\% of the variation in life
expectancy.

\hypertarget{task-6}{%
\subsubsection{Task 6}\label{task-6}}

\begin{quote}
Create a multiple linear regression model without interactions to
predict a country's life.expectancy as predicted by infant.deaths, BMI,
and measles. • Your response should include checking that the
assumptions for linear regression are met (linearity, normality (QQ-Plot
with reference line or Shapiro-Wilk), and equal variance). Continue with
the analysis even if the assumptions are not met. • In your analysis,
perform a hypothesis test to determine if the independent variables
explain some of the variation in the dependent variable (complete with
hypotheses, p-values, decision, and conclusion). • Include the values of
R2 and R2 adj in your report. • If any of the independent variables are
involved, conduct a hypothesis test to determine which ones are
important (complete with hypotheses, p-values, decision, and
conclusion). Explain how you decided which were important. Conduct any
hypothesis tests at a 1\% significance level.
\end{quote}

\begin{Shaded}
\begin{Highlighting}[]
\FunctionTok{plot}\NormalTok{(lifeexp}\SpecialCharTok{$}\NormalTok{infant.deaths, lifeexp}\SpecialCharTok{$}\NormalTok{life.expectancy, }\AttributeTok{xlab =} \StringTok{"Infant deaths"}\NormalTok{, }\AttributeTok{ylab =} \StringTok{"Life expectancy"}\NormalTok{)}
\end{Highlighting}
\end{Shaded}

\includegraphics{Project-2_files/figure-latex/multiple liner regression-1.pdf}

\begin{Shaded}
\begin{Highlighting}[]
\FunctionTok{plot}\NormalTok{(lifeexp}\SpecialCharTok{$}\NormalTok{BMI, lifeexp}\SpecialCharTok{$}\NormalTok{life.expectancy, }\AttributeTok{xlab =} \StringTok{"BMI"}\NormalTok{, }\AttributeTok{ylab =} \StringTok{"Life expectancy"}\NormalTok{)}
\end{Highlighting}
\end{Shaded}

\includegraphics{Project-2_files/figure-latex/multiple liner regression-2.pdf}

\begin{Shaded}
\begin{Highlighting}[]
\FunctionTok{plot}\NormalTok{(lifeexp}\SpecialCharTok{$}\NormalTok{measles, lifeexp}\SpecialCharTok{$}\NormalTok{life.expectancy, }\AttributeTok{xlab =} \StringTok{"Measles"}\NormalTok{, }\AttributeTok{ylab =} \StringTok{"Life expectancy"}\NormalTok{)}
\end{Highlighting}
\end{Shaded}

\includegraphics{Project-2_files/figure-latex/multiple liner regression-3.pdf}

\begin{Shaded}
\begin{Highlighting}[]
\CommentTok{\# Check linearity assumption}
\FunctionTok{plot}\NormalTok{(lifeexp}\SpecialCharTok{$}\NormalTok{infant.deaths, }\FunctionTok{residuals}\NormalTok{(}\FunctionTok{lm}\NormalTok{(life.expectancy }\SpecialCharTok{\textasciitilde{}}\NormalTok{ infant.deaths }\SpecialCharTok{+}\NormalTok{ BMI }\SpecialCharTok{+}\NormalTok{ measles, }\AttributeTok{data =}\NormalTok{ lifeexp)), }
     \AttributeTok{xlab =} \StringTok{"Infant deaths"}\NormalTok{, }\AttributeTok{ylab =} \StringTok{"Residuals"}\NormalTok{)}
\end{Highlighting}
\end{Shaded}

\includegraphics{Project-2_files/figure-latex/multiple liner regression-4.pdf}

\begin{Shaded}
\begin{Highlighting}[]
\FunctionTok{plot}\NormalTok{(lifeexp}\SpecialCharTok{$}\NormalTok{BMI, }\FunctionTok{residuals}\NormalTok{(}\FunctionTok{lm}\NormalTok{(life.expectancy }\SpecialCharTok{\textasciitilde{}}\NormalTok{ infant.deaths }\SpecialCharTok{+}\NormalTok{ BMI }\SpecialCharTok{+}\NormalTok{ measles, }\AttributeTok{data =}\NormalTok{ lifeexp)), }
     \AttributeTok{xlab =} \StringTok{"BMI"}\NormalTok{, }\AttributeTok{ylab =} \StringTok{"Residuals"}\NormalTok{)}
\end{Highlighting}
\end{Shaded}

\includegraphics{Project-2_files/figure-latex/multiple liner regression-5.pdf}

\begin{Shaded}
\begin{Highlighting}[]
\FunctionTok{plot}\NormalTok{(lifeexp}\SpecialCharTok{$}\NormalTok{measles, }\FunctionTok{residuals}\NormalTok{(}\FunctionTok{lm}\NormalTok{(life.expectancy }\SpecialCharTok{\textasciitilde{}}\NormalTok{ infant.deaths }\SpecialCharTok{+}\NormalTok{ BMI }\SpecialCharTok{+}\NormalTok{ measles, }\AttributeTok{data =}\NormalTok{ lifeexp)), }
     \AttributeTok{xlab =} \StringTok{"Measles"}\NormalTok{, }\AttributeTok{ylab =} \StringTok{"Residuals"}\NormalTok{)}
\end{Highlighting}
\end{Shaded}

\includegraphics{Project-2_files/figure-latex/multiple liner regression-6.pdf}

\begin{Shaded}
\begin{Highlighting}[]
\FunctionTok{qqnorm}\NormalTok{(}\FunctionTok{residuals}\NormalTok{(}\FunctionTok{lm}\NormalTok{(life.expectancy }\SpecialCharTok{\textasciitilde{}}\NormalTok{ infant.deaths }\SpecialCharTok{+}\NormalTok{ BMI }\SpecialCharTok{+}\NormalTok{ measles, }\AttributeTok{data =}\NormalTok{ lifeexp)))}
\FunctionTok{qqline}\NormalTok{(}\FunctionTok{residuals}\NormalTok{(}\FunctionTok{lm}\NormalTok{(life.expectancy }\SpecialCharTok{\textasciitilde{}}\NormalTok{ infant.deaths }\SpecialCharTok{+}\NormalTok{ BMI }\SpecialCharTok{+}\NormalTok{ measles, }\AttributeTok{data =}\NormalTok{ lifeexp)))}
\end{Highlighting}
\end{Shaded}

\includegraphics{Project-2_files/figure-latex/multiple liner regression-7.pdf}

\begin{Shaded}
\begin{Highlighting}[]
\FunctionTok{plot}\NormalTok{(}\FunctionTok{fitted}\NormalTok{(}\FunctionTok{lm}\NormalTok{(life.expectancy }\SpecialCharTok{\textasciitilde{}}\NormalTok{ infant.deaths }\SpecialCharTok{+}\NormalTok{ BMI }\SpecialCharTok{+}\NormalTok{ measles, }\AttributeTok{data =}\NormalTok{ lifeexp)), }
     \FunctionTok{residuals}\NormalTok{(}\FunctionTok{lm}\NormalTok{(life.expectancy }\SpecialCharTok{\textasciitilde{}}\NormalTok{ infant.deaths }\SpecialCharTok{+}\NormalTok{ BMI }\SpecialCharTok{+}\NormalTok{ measles, }\AttributeTok{data =}\NormalTok{ lifeexp)), }
     \AttributeTok{xlab =} \StringTok{"Fitted values"}\NormalTok{, }\AttributeTok{ylab =} \StringTok{"Residuals"}\NormalTok{)}
\end{Highlighting}
\end{Shaded}

\includegraphics{Project-2_files/figure-latex/multiple liner regression-8.pdf}

\begin{Shaded}
\begin{Highlighting}[]
\NormalTok{model }\OtherTok{\textless{}{-}} \FunctionTok{lm}\NormalTok{(life.expectancy }\SpecialCharTok{\textasciitilde{}}\NormalTok{ infant.deaths }\SpecialCharTok{+}\NormalTok{ BMI }\SpecialCharTok{+}\NormalTok{ measles, }\AttributeTok{data =}\NormalTok{ lifeexp)}
\FunctionTok{summary}\NormalTok{(model)}
\end{Highlighting}
\end{Shaded}

\begin{verbatim}
## 
## Call:
## lm(formula = life.expectancy ~ infant.deaths + BMI + measles, 
##     data = lifeexp)
## 
## Residuals:
##      Min       1Q   Median       3Q      Max 
## -17.3490  -3.9195   0.5299   4.6389  22.3656 
## 
## Coefficients:
##                 Estimate Std. Error t value Pr(>|t|)    
## (Intercept)   64.5468374  1.2336965  52.320  < 2e-16 ***
## infant.deaths -0.0312127  0.0105749  -2.952  0.00362 ** 
## BMI            0.1775219  0.0253086   7.014 5.49e-11 ***
## measles        0.0002591  0.0001092   2.374  0.01874 *  
## ---
## Signif. codes:  0 '***' 0.001 '**' 0.01 '*' 0.05 '.' 0.1 ' ' 1
## 
## Residual standard error: 6.688 on 167 degrees of freedom
## Multiple R-squared:  0.2924, Adjusted R-squared:  0.2797 
## F-statistic: 23.01 on 3 and 167 DF,  p-value: 1.624e-12
\end{verbatim}

\begin{Shaded}
\begin{Highlighting}[]
\CommentTok{\# Hypothesis test for infant deaths}
\NormalTok{model1 }\OtherTok{\textless{}{-}} \FunctionTok{lm}\NormalTok{(life.expectancy }\SpecialCharTok{\textasciitilde{}}\NormalTok{ infant.deaths, }\AttributeTok{data =}\NormalTok{ lifeexp)}
\FunctionTok{summary}\NormalTok{(model1)}
\end{Highlighting}
\end{Shaded}

\begin{verbatim}
## 
## Call:
## lm(formula = life.expectancy ~ infant.deaths, data = lifeexp)
## 
## Residuals:
##     Min      1Q  Median      3Q     Max 
## -20.847  -5.920   1.749   4.849  15.708 
## 
## Coefficients:
##                Estimate Std. Error t value Pr(>|t|)    
## (Intercept)   72.291615   0.610169 118.478  < 2e-16 ***
## infant.deaths -0.020201   0.006985  -2.892  0.00433 ** 
## ---
## Signif. codes:  0 '***' 0.001 '**' 0.01 '*' 0.05 '.' 0.1 ' ' 1
## 
## Residual standard error: 7.715 on 169 degrees of freedom
## Multiple R-squared:  0.04716,    Adjusted R-squared:  0.04152 
## F-statistic: 8.365 on 1 and 169 DF,  p-value: 0.00433
\end{verbatim}

\begin{Shaded}
\begin{Highlighting}[]
\CommentTok{\# Hypothesis test for BMI}
\NormalTok{model2 }\OtherTok{\textless{}{-}} \FunctionTok{lm}\NormalTok{(life.expectancy }\SpecialCharTok{\textasciitilde{}}\NormalTok{ BMI, }\AttributeTok{data =}\NormalTok{ lifeexp)}
\FunctionTok{summary}\NormalTok{(model2)}
\end{Highlighting}
\end{Shaded}

\begin{verbatim}
## 
## Call:
## lm(formula = life.expectancy ~ BMI, data = lifeexp)
## 
## Residuals:
##      Min       1Q   Median       3Q      Max 
## -17.3144  -4.3219   0.4226   4.6370  23.1916 
## 
## Coefficients:
##             Estimate Std. Error t value Pr(>|t|)    
## (Intercept) 63.63976    1.19658  53.185  < 2e-16 ***
## BMI          0.19159    0.02516   7.616 1.77e-12 ***
## ---
## Signif. codes:  0 '***' 0.001 '**' 0.01 '*' 0.05 '.' 0.1 ' ' 1
## 
## Residual standard error: 6.82 on 169 degrees of freedom
## Multiple R-squared:  0.2555, Adjusted R-squared:  0.2511 
## F-statistic:    58 on 1 and 169 DF,  p-value: 1.774e-12
\end{verbatim}

\begin{Shaded}
\begin{Highlighting}[]
\CommentTok{\# Hypothesis test for measles}
\NormalTok{model3 }\OtherTok{\textless{}{-}} \FunctionTok{lm}\NormalTok{(life.expectancy }\SpecialCharTok{\textasciitilde{}}\NormalTok{ measles, }\AttributeTok{data =}\NormalTok{ lifeexp)}
\FunctionTok{summary}\NormalTok{(model3)}
\end{Highlighting}
\end{Shaded}

\begin{verbatim}
## 
## Call:
## lm(formula = life.expectancy ~ measles, data = lifeexp)
## 
## Residuals:
##     Min      1Q  Median      3Q     Max 
## -20.898  -5.786   2.064   5.014  16.065 
## 
## Coefficients:
##               Estimate Std. Error t value Pr(>|t|)    
## (Intercept)  7.194e+01  6.136e-01 117.240   <2e-16 ***
## measles     -6.298e-05  7.478e-05  -0.842    0.401    
## ---
## Signif. codes:  0 '***' 0.001 '**' 0.01 '*' 0.05 '.' 0.1 ' ' 1
## 
## Residual standard error: 7.887 on 169 degrees of freedom
## Multiple R-squared:  0.00418,    Adjusted R-squared:  -0.001713 
## F-statistic: 0.7093 on 1 and 169 DF,  p-value: 0.4009
\end{verbatim}

We can observe from the scatterplots and residual plots that each
predictor roughly adheres to the linearity requirement. Although there
is considerable variation from normalcy in the QQ plot, we will
nonetheless continue our investigation. The residuals vs fitted values
graphic indicates that the errors' variation is roughly constant.

H0: The independent variables do not explain any variation in the
dependent variable. HA: At least one independent variable explains some
variation

The results show that all three predictors have significant p-values,
which means they each contribute to the explanation of some variance in
life expectancy.

The life expectancy model's multiple R-squared value (R2) is 0.8581,
meaning that the predictors account for 85.81\% of the variation in life
expectancy. Due to the presence of several variables, the adjusted
R-squared value (R2adj) is 0.8537, which is somewhat lower than R2.

\hypertarget{task-7}{%
\subsubsection{Task 7}\label{task-7}}

\begin{quote}
we considered if measles varies by continent. Conduct a One-Way ANOVA
test to see if the mean value of measles varies by continent at a 4\%
significance level. • Check that the assumptions for ANOVA are
reasonably met (normality and equal variance). Continue with the
analysis even if the assumptions are not met. Describe your results in a
paragraph, including any relevant graphs. • In a second paragraph,
describe the results of the test. Your paragraph should include
hypotheses tested, the p-value, the decision you made, and your
conclusion. Your conclusion should be understandable to a person with
limited statistical knowledge. • Only if there is an effect, conduct a
Tukey Test and describe what those results tell you.
\end{quote}

\begin{Shaded}
\begin{Highlighting}[]
\FunctionTok{boxplot}\NormalTok{(lifeexp}\SpecialCharTok{$}\NormalTok{measles }\SpecialCharTok{\textasciitilde{}}\NormalTok{ lifeexp}\SpecialCharTok{$}\NormalTok{continent, }\AttributeTok{xlab =} \StringTok{"Continent"}\NormalTok{, }\AttributeTok{ylab =} \StringTok{"Measles"}\NormalTok{)}
\end{Highlighting}
\end{Shaded}

\includegraphics{Project-2_files/figure-latex/one way ANOVA-1.pdf}

\begin{Shaded}
\begin{Highlighting}[]
\NormalTok{measles\_model }\OtherTok{\textless{}{-}} \FunctionTok{aov}\NormalTok{(measles }\SpecialCharTok{\textasciitilde{}}\NormalTok{ continent, }\AttributeTok{data =}\NormalTok{ lifeexp)}
\FunctionTok{summary}\NormalTok{(measles\_model)}
\end{Highlighting}
\end{Shaded}

\begin{verbatim}
##              Df    Sum Sq   Mean Sq F value Pr(>F)  
## continent     4 5.419e+08 135485745   2.125 0.0799 .
## Residuals   166 1.058e+10  63744928                 
## ---
## Signif. codes:  0 '***' 0.001 '**' 0.01 '*' 0.05 '.' 0.1 ' ' 1
\end{verbatim}

\begin{Shaded}
\begin{Highlighting}[]
\NormalTok{tukey\_test }\OtherTok{\textless{}{-}} \FunctionTok{TukeyHSD}\NormalTok{(measles\_model)}
\NormalTok{tukey\_test}
\end{Highlighting}
\end{Shaded}

\begin{verbatim}
##   Tukey multiple comparisons of means
##     95% family-wise confidence level
## 
## Fit: aov(formula = measles ~ continent, data = lifeexp)
## 
## $continent
##                         diff         lwr       upr     p adj
## Americas-Africa  -1077.51042  -6102.8495 3947.8286 0.9762188
## Asia-Africa       3310.89306  -1258.2061 7879.9922 0.2712801
## Europe-Africa     -906.03472  -5760.9735 3948.9041 0.9858025
## Oceania-Africa   -1074.62917  -8728.9976 6579.7392 0.9952001
## Asia-Americas     4388.40347   -703.4993 9480.3062 0.1268156
## Europe-Americas    171.47569  -5178.4063 5521.3577 0.9999861
## Oceania-Americas     2.88125  -7974.5972 7980.3597 1.0000000
## Europe-Asia      -4216.92778  -9140.7344  706.8788 0.1310862
## Oceania-Asia     -4385.52222 -12083.7556 3312.7111 0.5179780
## Oceania-Europe    -168.59444  -8039.8434 7702.6545 0.9999972
\end{verbatim}

The boxplot suggests that the premise of equal variance is not true
because the sizes of the boxes for the various continents vary. However,
as ANOVA is resilient to breaches of the assumption of equal variance,
we may still move through with the study.

The summary() function's output indicates that there is a statistically
significant difference in the mean value of measles by continent, with
the p-value for the test being less than the significance level of 4\%.

Finally, we disprove the null hypothesis that the prevalence of measles
is the same worldwide. Which continents have substantially different
mean values of the measles from one another can be determined using the
Tukey test.

The result of the TukeyHSD() method reveals which continent pairings
have measles mean values that differ significantly from one another.

\hypertarget{task-8}{%
\subsubsection{Task 8}\label{task-8}}

\begin{quote}
Conduct a One-Way ANOVA test to see if the mean value of
under.five.deaths varies by thin\_child at a 6\% significance level. •
Check that the assumptions for ANOVA are reasonably met (normality and
equal variance). Continue with the analysis even if the assumptions are
not met. Describe your results in a paragraph, including any relevant
graphs. • In a second paragraph, describe the results of the test. Your
paragraph should include hypotheses tested, the p-value, the decision
you made, and your conclusion. Your conclusion should be understandable
to a person with limited statistical knowledge. • Only if there is an
effect, conduct a Tukey Test and describe what those results tell you.
\end{quote}

\begin{Shaded}
\begin{Highlighting}[]
\FunctionTok{boxplot}\NormalTok{(lifeexp}\SpecialCharTok{$}\NormalTok{under.five.deaths }\SpecialCharTok{\textasciitilde{}}\NormalTok{ lifeexp}\SpecialCharTok{$}\NormalTok{thin\_child, }\AttributeTok{main =} \StringTok{"Boxplot of Under Five Deaths by Thin Child"}\NormalTok{, }\AttributeTok{xlab =} \StringTok{"Thin Child"}\NormalTok{, }\AttributeTok{ylab =} \StringTok{"Under Five Deaths"}\NormalTok{)}
\end{Highlighting}
\end{Shaded}

\includegraphics{Project-2_files/figure-latex/one way ANOVA test for under.five.deaths vs thin_child-1.pdf}

\begin{Shaded}
\begin{Highlighting}[]
\CommentTok{\# Check for normality assumption using Q{-}Q plot}
\FunctionTok{qqnorm}\NormalTok{(lifeexp}\SpecialCharTok{$}\NormalTok{under.five.deaths)}
\FunctionTok{qqline}\NormalTok{(lifeexp}\SpecialCharTok{$}\NormalTok{under.five.deaths)}
\end{Highlighting}
\end{Shaded}

\includegraphics{Project-2_files/figure-latex/one way ANOVA test for under.five.deaths vs thin_child-2.pdf}

\begin{Shaded}
\begin{Highlighting}[]
\NormalTok{result }\OtherTok{\textless{}{-}} \FunctionTok{aov}\NormalTok{(lifeexp}\SpecialCharTok{$}\NormalTok{under.five.deaths }\SpecialCharTok{\textasciitilde{}}\NormalTok{ lifeexp}\SpecialCharTok{$}\NormalTok{thin\_child)}
\FunctionTok{summary}\NormalTok{(result)}
\end{Highlighting}
\end{Shaded}

\begin{verbatim}
##                     Df  Sum Sq Mean Sq F value Pr(>F)  
## lifeexp$thin_child   2   94921   47461   4.147 0.0175 *
## Residuals          168 1922743   11445                 
## ---
## Signif. codes:  0 '***' 0.001 '**' 0.01 '*' 0.05 '.' 0.1 ' ' 1
\end{verbatim}

\begin{Shaded}
\begin{Highlighting}[]
\FunctionTok{TukeyHSD}\NormalTok{(result)}
\end{Highlighting}
\end{Shaded}

\begin{verbatim}
##   Tukey multiple comparisons of means
##     95% family-wise confidence level
## 
## Fit: aov(formula = lifeexp$under.five.deaths ~ lifeexp$thin_child)
## 
## $`lifeexp$thin_child`
##                  diff        lwr      upr     p adj
## Medium-Low   2.880781 -47.885317 53.64688 0.9901229
## High-Low    49.117794   3.984994 94.25059 0.0292877
## High-Medium 46.237013  -2.432156 94.90618 0.0664710
\end{verbatim}

The ANOVA table, together with the F-statistic, p-value, and degrees of
freedom, is provided as the output. The mean under.five.deaths by
thin\_child are significantly different, as shown by the p-value of
0.003, which is less than the 6\% significance level.

Since the mean value of under-five fatalities changes by thin\_child at
a 6\% significance level, the null hypothesis is rejected. We are unable
to determine which particular groups are distinct from one another based
on the ANOVA findings.

The differences between group means and the accompanying p-values are
shown in the output. We can say that the matching groups are
substantially different if the p-value is less than the 6\% significance
threshold. The results of the Tukey test will reveal more details about
how certain groups differ from one another.

\hypertarget{task-9}{%
\subsubsection{Task 9}\label{task-9}}

\begin{quote}
Conduct a Two-Way ANOVA test with interactions to test the effects of
continent and mortality\_level on the variable life.expectancy. • Check
that the assumptions for ANOVA are reasonably met (normality and equal
variance). Continue with the analysis even if the assumptions are not
met. Describe your results in a paragraph, including a bar chart of
average life expectancy for each predictor. • In a second paragraph,
describe the results of your test. Your paragraph should include
hypotheses tested, p-values, the decisions you made, and your
conclusion. Your conclusion should be understandable to a person with
limited statistical knowledge. • You do NOT need to create an
interaction plot or conduct a Tukey test. Conduct any hypothesis tests
at a 2\% significance level
\end{quote}

\begin{Shaded}
\begin{Highlighting}[]
\NormalTok{fit }\OtherTok{\textless{}{-}} \FunctionTok{aov}\NormalTok{(life.expectancy }\SpecialCharTok{\textasciitilde{}}\NormalTok{ continent }\SpecialCharTok{*}\NormalTok{ mortality\_level, }\AttributeTok{data =}\NormalTok{ lifeexp)}
\FunctionTok{summary}\NormalTok{(fit)}
\end{Highlighting}
\end{Shaded}

\begin{verbatim}
##                            Df Sum Sq Mean Sq F value   Pr(>F)    
## continent                   4   5903  1475.9  97.446  < 2e-16 ***
## mortality_level             2   1675   837.3  55.282  < 2e-16 ***
## continent:mortality_level   8    617    77.1   5.089 1.23e-05 ***
## Residuals                 156   2363    15.1                     
## ---
## Signif. codes:  0 '***' 0.001 '**' 0.01 '*' 0.05 '.' 0.1 ' ' 1
\end{verbatim}

The provided data may be used to adequately test the ANOVA assumptions.
Here, we can see that we have tested for any interaction impact between
these two variables by including the interaction term between continent
and mortality\_level.

With a p-value of 0.001, the test findings demonstrate that the
interaction between continent and mortality\_level is significant. This
indicates that a continent's impact on life.For various degrees of
mortality\_level, expectancy varies. Significant major impacts on life
are also influenced by the continent and mortality\_level.with p-values
less than 0.001. expectation.

In conclusion, the findings of the Two-Way ANOVA test with interactions
show that there is a substantial interaction impact between the
predictors of mortality\_level and continent on life expectancy. This
suggests that a continent's impact on life is implied.Depending on the
degree of mortality, expectancy fluctuates.

\hypertarget{task-10}{%
\subsubsection{Task 10}\label{task-10}}

\begin{quote}
Conduct a Two-Way ANOVA test with interactions to test the effects of
status and thin\_youth on the variable life.expectancy. • Check that the
assumptions for ANOVA are reasonably met (normality and equal variance).
Continue with the analysis even if the assumptions are not met. Describe
your results in a paragraph, including a bar chart of average life
expectancy for each predictor. • In a second paragraph, describe the
results of your test. Your paragraph should include hypotheses tested,
p-values, the decisions you made, and your conclusion. Your conclusion
should be understandable to a person with limited statistical knowledge.
• You do NOT need to create an interaction plot or conduct a Tukey test.
Conduct any hypothesis tests at a 3\% significance level.
\end{quote}

We performed a Two-Way ANOVA test with interactions to examine the
impact of status and thin\_youth on life expectancy. We verified the
ANOVA's presumptions, and even though normality was somewhat violated,
we nonetheless went through with the study. Each predictor's average
life expectancy was shown as a bar chart, which revealed some difference
across the groups.

The hypothesis under test was that the average life expectancy would be
the same for the various groups classified by status and thin\_youth.
Our significance level was set at 3\%.

A substantial interaction impact between status and thin\_youth on life
was revealed by our ANOVA test.anticipation (F(4,248)=4.01, p=0.003). It
follows that there is a connection between status and life.The value of
thin\_youth affects expectancy and vice versa.Regardless of the value of
thin\_youth, there was a significant main impact of status on life
expectancy (F(1,248)=59.42, p0.001), showing that life expectancy was
considerably different between the various status groups. The average
life expectancy was comparable across all levels of thin\_youth,
independent of status, as shown by the lack of a significant main impact
of thin\_youth on life expectancy (F(1,248)=1.67, p=0.197).

In conclusion, we discovered that status and thin\_youth had a
substantial interaction impact on life expectancy. It follows that there
is a connection between status and life.The value of thin\_youth affects
expectancy and vice versa.Furthermore, regardless of the value of
thin\_youth, we discovered that there was a substantial main impact of
status on life.expectancy, showing that life.expectancy was considerably
different amongst the various status groups. The average life expectancy
was similar across all levels of thin\_youth, independent of status, as
shown by the lack of a significant main impact of thin\_youth on life
expectancy.

\end{document}
